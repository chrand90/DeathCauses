\documentclass[a4paper, 12pt]{memoir}
\usepackage[danish]{babel}
\usepackage[utf8]{inputenc}
\renewcommand{\danishhyphenmins}{22}
\usepackage[T1]{fontenc}
\usepackage{amsmath}
\usepackage{amssymb}
\newcommand{\Cov}{\textup{Cov}}
\newcommand{\xdot}{x_{\cdot}}
\usepackage{SASnRdisplay}
\usepackage{mathtools}
\newcommand{\Rroom}[2]{\mathbb R^{#1 \times #2}}
\renewcommand{\i}{^{-1}}
\newcommand{\tr}{\textup{tr}}
\usepackage{bm}
\usepackage{tikz}
\usepackage[normalem]{ulem}
\definecolor{blue1}{rgb}{0.7,0.8,1}
\definecolor{blue2}{rgb}{0.6,0.8,1}
\definecolor{blue3}{rgb}{0.7,0.9,1}
\definecolor{blue4}{rgb}{0.6,0.7,1}

\definecolor{red1}{rgb}{1,0.5,0.7}
\definecolor{red2}{rgb}{1,0.7,0.75}
\definecolor{red3}{rgb}{1,0.7,0.5}
\definecolor{red4}{rgb}{1,0.8,0.7}
\definecolor{red5}{rgb}{1,0.5,0.5}

\definecolor{gron}{rgb}{0,0.6,0}

\newcommand{\bigzero}{\mbox{\normalfont\Large\bfseries 0}}
\newcommand{\rvline}{\hspace*{-\arraycolsep}\vline\hspace*{-\arraycolsep}}



\begin{document}

Lad os kigge på formlen fra notation and integration.pdf

\begin{equation}
U(F_I)=p_d(a,\psi_{0,F_I})-p_d(a, \psi_0)
\end{equation}
Begge led i summen kan skrives som produkter
\begin{align}
&p_d(a,\psi_{0,F_I})-p_d(a, \psi_0)\\
&=P_d(a) \biggl(\begin{lgathered}[t]\prod_{j=1}^{n_d} \frac{g_{j,d}(R^{1,j,d}(\psi_{0,F_I}^{1,j,d}),  \dots , R^{k_{j,d},j,d}(\psi_{0,F_I}^{k_{j,d},j,d}) )}{\textup{norm}^{j,d}(l(a))}-\\
 \prod_{j=1}^{n_d} \frac{g_{j,d}(R^{1,j,d}(\psi_{0}^{1,j,d}),  \dots , R^{k_{j,d},j,d}(\psi_{0}^{k_{j,d},j,d}) )}{\textup{norm}^{j,d}(l(a))}\biggr)\end{lgathered}\\
 &=P_d(a) \frac{1}{\prod_{j=1}^{n_d}{\textup{norm}^{j,d}(l(a))}}\biggl(\begin{lgathered}[t]\prod_{j=1}^{n_d} {g_{j,d}(R^{1,j,d}(\psi_{0,F_I}^{1,j,d}),  \dots , R^{k_{j,d},j,d}(\psi_{0,F_I}^{k_{j,d},j,d}) )}-\\
 \prod_{j=1}^{n_d} {g_{j,d}(R^{1,j,d}(\psi_{0}^{1,j,d}),  \dots , R^{k_{j,d},j,d}(\psi_{0}^{k_{j,d},j,d}) )}\biggr)\end{lgathered}\\
\end{align}
Hvis vi antager at $g$'erne er produkter, står der
\begin{align}
 &=P_d(a) \frac{1}{\prod_{j=1}^{n_d}{\textup{norm}^{j,d}(l(a))}}\biggl(\begin{lgathered}[t]\prod_{j=1}^{n_d} {R^{1,j,d}(\psi_{0,F_I}^{1,j,d})  \cdots R^{k_{j,d},j,d}(\psi_{0,F_I}^{k_{j,d},j,d}) }-\\
 \prod_{j=1}^{n_d} {R^{1,j,d}(\psi_{0}^{1,j,d}) \cdots R^{k_{j,d},j,d}(\psi_{0}^{k_{j,d},j,d}) }\biggr)\end{lgathered}\\
\end{align}
Inde i den store parentes står der et objekt af formen
\begin{align}
r_1r_2\cdots r_n- r_{1,0}r_{2,0}\cdots r_{n,0}
\end{align}
Det kan skrives som
\begin{align}
(r_{1,0}&+(r_1-r_{1,0}))\cdots (r_{n,0}+(r_n-r_{n,0}))- r_{1,0}r_{2,0}\cdots r_{n,0}\\
&=r_{1,0}\cdots r_{n,0}+(r_1-r_{1,0})r_{2,0}\cdots r_{n,0}+\dots + (r_n-r_{n,0})r_{1,0}\cdots r_{n-1,0}+ R -r_{1,0}\cdots r_{n,0}\\
&=(r_1-r_{1,0})r_{2,0}\cdots r_{n,0}+\dots + (r_n-r_{n,0})r_{1,0}\cdots r_{n-1,0}+ R
\end{align}
Vi vil nu dele $R$ ud på risk factorne svarende til $r_1$, $r_2 \dots , r_n$. Formlen i notation and integration.pdf ser således ud
\begin{equation}
S^{(1)}(F_i)=\frac{S(F_i)}{\sum_{j=1}^{n}S(F_j)}\cdot U_* \label{eqS1}
\end{equation}
hvilket siger at den ekstra vægt (dvs. $U_{*}/(\sum_{j=1}^{n}S(F_j))$) skal \textcolor{red}{\sout{deles ud proportionalt til $r_i/\textup{norm}^{i}$}} \textcolor{blue}{deles ud proportionalt til 
\begin{equation}
r_{1,0}\cdots r_{i-1,0}\cdot (r_i-r_{i,0}) \cdot r_{i+1,0}\cdots r_n
\end{equation}
hvilket er upåvirket af omskaleringer af hele risk ratiotabeller.
}

For klarifikation kan vi skrive
\begin{align}
 &P_d(a) \frac{1}{\prod_{j=1}^{n_d}{\textup{norm}^{j,d}(l(a))}}\biggl(\begin{lgathered}[t]\prod_{j=1}^{n_d} {R^{1,j,d}(\psi_{0,F_I}^{1,j,d})  \cdots R^{k_{j,d},j,d}(\psi_{0,F_I}^{k_{j,d},j,d}) }-\\
 \prod_{j=1}^{n_d} {R^{1,j,d}(\psi_{0}^{1,j,d}) \cdots R^{k_{j,d},j,d}(\psi_{0}^{k_{j,d},j,d}) }\biggr)\end{lgathered}\\
 &=\sum_{j=1}^{n}S(F_i)+P_d(a) \frac{1}{\prod_{j=1}^{n_d}{\textup{norm}^{j,d}(l(a))}}\biggl(R\biggr)=U_{*}
\end{align}
På branchen prototype brugte vi \textcolor{red}{\sout{i stedet for}} \textcolor{blue}{denne udgave af} formel \eqref{eqS1}
\begin{equation}
S^{(1)}(F_j)=\frac{\frac{r_j-r_{j,0}}{r_{j,0}}}{\sum_{i=1}^n\frac{r_i-r_{i,0}}{r_{i,0}}} U_{*}
\end{equation}

\subsection{Generelt}
Lad som sædvanlig $U(F_I)=p_d(a,\psi_{F_I})-p_d(a, \psi_{0})$ og $U_*=U(F_{\{1,\dots,N\}})$. Vi ønsker at lave dekomponeringen 
\begin{equation}
U_*=\sum_{i}S(F_i)
\end{equation}
og vi ønsker at $S(F_i)$ på en eller anden måde er tæt på $U(F_i)$. I multinomialfordelingen ville det fortolkes som at produktet
\begin{equation}
\prod_{i=1}^{N}\bigl(S(F_i)\bigr)^{U(F_i)}
\end{equation}
er stort. Man kan vise at 
\begin{gather}
\textup{arg}\max_{S(F_1), \dots, S(F_N) \in \mathbb R_+} \prod_{i=1}^{N}\bigl(S(F_i)\bigr)^{U(F_i)}, 	\\\quad \textup{ med betingelsen } \sum_{i=1}^NS(F_i)=U_*
\end{gather}
giver løsningen 
\begin{equation}
\S(F_i)=\frac{U(F_i)}{\sum_{j=1}^N U(F_j)}U_*
\end{equation}
hvilket er identisk med formel \eqref{eqS1}. 








\end{document}